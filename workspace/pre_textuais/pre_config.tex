%==================================================================================
%                                    CAPA
%==================================================================================
\printmodelo           % Imprime o nome modelo na capa. Comente isto posteriormente.
\capa{logotipo=black}  % Gera o modelo da capa externa do trabalho, recebe 3 argumentos:
%                          1) vazio (sem espacos): imprime o titulo apenas em texto;
%                          2) logotipo: imprime o titulo com o logotipo colorido; 
%                          3) logotipo=black: imprime o titulo com o logotipo na cor preta.

%==================================================================================
%                    AUTORIZACAO PARA PUBLICACAO NA INTERNET
%==================================================================================
% Gera a autorizacao para publicacao em formato eletronico:

\publicacao

%==================================================================================
%                               FOLHA DE ROSTO
%==================================================================================
% Gera a folha de rosto do trabalho:

\folhaderosto

%==================================================================================
%                       FICHA CATALOGRAFICA DA BIBLIOTECA
%==================================================================================
% Gera um modelo de ficha catalografia, que deve ser substituido pelo modelo
% fornecido pela biblioteca ou orgao responsavel:

\cdu{}

%==================================================================================
%                              MEMBROS DA BANCA
%==================================================================================
% Se o co-orientador estiver presente na banca, insira o nome dele como se fosse
% apenas um outro membro

\banca*            % Inclui o nome do(a) orientador(a) automaticamente como presidente da banca
\banca{\textbf{NOME DO MEMBRO DA BANCA} \\ FACULDADE DO MEMBRO DA BANCA }
\banca{\textbf{ } }
\bancavskip{40pt}   % Define o espaçamento vertical entre os nomes dos membros da banca
\bancahrule{10cm}   % Define o comprimento da linha horizontal sobre o nome dos membros da banca

%==================================================================================
%                             TERMO DE APROVACAO
%==================================================================================
% Gera o termo de aprovacao:

\aprovacao

%==================================================================================
%                             DIREITOS AUTORAIS
%==================================================================================
%==================================================================================
%                             DIREITOS AUTORAIS
%==================================================================================
% Ambiente para escrever um texto resumido do perfil do autor na pagina de
% direitos autorais:

\begin{perfil}
Texto com um perfil resumido do autor do trabalho. Por exemplo: Graduou--se em Física (bacharelado) na Universidade Federal de Goiás (UFG) em 2013. Durante sua graduação, desenvolveu um trabalho de monografia em Computação Quântica. Em 2014, ingressou no mestrado em Ciência da Computação no Instituto de Informática (UFG). Durante o mestrado, foi bolsista CNPq e desenvolveu um trabalho teórico em predição de estruturas terciárias de proteínas.
\end{perfil}

%==================================================================================
%                               DEDICATÓRIA
%==================================================================================
%==================================================================================
%                               DEDICATÓRIA
%==================================================================================
% Ambiente para escrever a dedicatoria do trabalho:

\begin{dedicatoria}
Dedicatória do trabalho a alguma pessoa, entidade, etc.
\end{dedicatoria}

%==================================================================================
%                             AGRADECIMENTOS
%==================================================================================
%==================================================================================
%                             AGRADECIMENTOS
%==================================================================================
% Ambiente para escrever os agradecimentos:

\begin{agradecimentos}
Texto com agradecimentos àquelas pessoas/entidades que, na opinião do autor, deram alguma contribuição relevante para o desenvolvimento do trabalho.
\end{agradecimentos}

%==================================================================================
%                               EPIGRAFE
%==================================================================================
%==================================================================================
%                               EPIGRAFE
%==================================================================================
% Ambiente para escrever a epigrafe:

\begin{epigrafe}{Prof. Dr. Salviano de Araújo Leão}
\textit{``Veja bem, é muito simples, não é difícil, é trabalhoso!''}
\end{epigrafe}

%==================================================================================
%                                RESUMO
%==================================================================================
%==================================================================================
%                                RESUMO
%==================================================================================
% Ambiente para escrever o resumo do trabalho:

\begin{resumo}{pt}{Classe. \LaTeX. Formatação.}
Esta classe \LaTeX\ foi desenvolvida para a formatação de trabalhos acadêmicos do Instituto de Física da Universidade Federal de Goiás e atualmente suporta os seguintes tipos: trabalho de conclusão de curso (TCC), monografia, qualificação de mestrado, dissertação de mestrado e tese de doutorado.
\end{resumo}

\begin{resumo}{en}{Class. \LaTeX. Formatting.}
This \LaTeX\ class is designed for the academic writing format of the Institute of Physics of the Federal University of Goiás and currently supports the following types: completion of course work (TCC), monograph, master's qualification, master's thesis and doctoral dissertation.
\end{resumo}

%==================================================================================
%                       LISTA DE FIGURAS E TABELAS
%==================================================================================
% Apenas ative as listas caso existam figuras e/ou tabelas no seu documento. 
% Figuras e tabelas sem legenda (caption), nao aparecem por padrao nas listas.

\listadefiguras
\listadetabelas

%==================================================================================
%                      LISTA DE CODIGOS E ALGORITMOS
%==================================================================================
% Apenas ative as listas caso existam codigos e/ou algoritmos no seu documento.:

% \listadealgoritmos
% \listadecodigos

%==================================================================================
%                                SUMARIO
%==================================================================================
% Gera o sumario do trabalho:

\sumario